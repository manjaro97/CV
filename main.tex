%% Use the "normalphoto" option if you want a normal photo instead of cropped to a circle
% \documentclass[10pt,a4paper,normalphoto]{altacv}

\documentclass[10pt,a4paper,ragged2e,withhyper]{altacv}
%% AltaCV uses the fontawesome5 and simpleicons packages.
%% See http://texdoc.net/pkg/fontawesome5 and http://texdoc.net/pkg/simpleicons for full list of symbols.

% Change the page layout if you need to
\geometry{left=1.25cm,right=1.25cm,top=1.5cm,bottom=1.5cm,columnsep=1.2cm}

% The paracol package lets you typeset columns of text in parallel
\usepackage{paracol}


% Change the font if you want to, depending on whether
% you're using pdflatex or xelatex/lualatex
% WHEN COMPILING WITH XELATEX PLEASE USE
% xelatex -shell-escape -output-driver="xdvipdfmx -z 0" mmayer.tex
\ifxetexorluatex
  % If using xelatex or lualatex:
  \setmainfont{Lato}
\else
  % If using pdflatex:
  \usepackage[default]{lato}
\fi

% Change the colours if you want to
\definecolor{NavyBlue}{HTML}{0a173b}
\definecolor{SlateGrey}{HTML}{2E2E2E}
\definecolor{LightGrey}{HTML}{666666}
% \colorlet{name}{black}
% \colorlet{tagline}{PastelRed}
\colorlet{heading}{NavyBlue}
\colorlet{tagline}{NavyBlue}
\colorlet{headingrule}{NavyBlue}
% \colorlet{subheading}{PastelRed}
\colorlet{accent}{NavyBlue}
\colorlet{emphasis}{SlateGrey}
\colorlet{body}{LightGrey}

% Change some fonts, if necessary
% \renewcommand{\namefont}{\Huge\rmfamily\bfseries}
% \renewcommand{\personalinfofont}{\footnotesize}
% \renewcommand{\cvsectionfont}{\LARGE\rmfamily\bfseries}
% \renewcommand{\cvsubsectionfont}{\large\bfseries}

% Change the bullets for itemize and rating marker
% for \cvskill if you want to
\renewcommand{\cvItemMarker}{{\small\textbullet}}
\renewcommand{\cvRatingMarker}{\faCircle}
% ...and the markers for the date/location for \cvevent
\renewcommand{\cvDateMarker}{\faCalendar*[regular]}
\renewcommand{\cvLocationMarker}{}


% If your CV/résumé is in a language other than English,
% then you probably want to change these so that when you
% copy-paste from the PDF or run pdftotext, the location
% and date marker icons for \cvevent will paste as correct
% translations. For example Spanish:
% \renewcommand{\locationname}{Ubicación}
% \renewcommand{\datename}{Fecha}


%% Use (and optionally edit if necessary) this .tex if you
%% want to use an author-year reference style like APA(6)
%% for your publication list
% \input{pubs-authoryear.tex}

%% Use (and optionally edit if necessary) this .tex if you
%% want an originally numerical reference style like IEEE
%% for your publication list
\input{pubs-num.tex}


\begin{document}
\name{Jonas Carlsson}
\tagline{Civilingenjör i Datateknik / Mjukvaruutvecklare}
% Cropped to square from https://en.wikipedia.org/wiki/Marissa_Mayer#/media/File:Marissa_Mayer_May_2014_(cropped).jpg, CC-BY 2.0
%% You can add multiple photos on the left or right
\photoR{4.5cm}{profilbild}
% \photoL{2cm}{Yacht_High,Suitcase_High}
\personalinfo{%
  % Not all of these are required!
  % You can add your own with \printinfo{symbol}{detail}
  \email{j_c_kc@live.se}
  \phone{070-3813089}
  
  \vspace{0.2cm}
  
  \location{Timrå}
  \linkedin{jonas-carlsson-456b561a7}
%   \github{github.com/mmayer} % I'm just making this up though.
}

\makecvheader

%% Depending on your tastes, you may want to make fonts of itemize environments slightly smaller
\AtBeginEnvironment{itemize}{\small}

%% Set the left/right column width ratio to 6:4.
\columnratio{0.6}

% Start a 2-column paracol. Both the left and right columns will automatically
% break across pages if things get too long.
\begin{paracol}{2}

\cvsection{Arbetslivserfarenhet}

\cvevent{Arkitekt \& Utvecklare}{AdjustIT AB}{Maj 2025 -- Pågående}

\begin{itemize}
  \item Bygger en multitenant webbplats och en cross-platform app som är billig och lätt att använda för bland annat föreningar.
  \item Funktionalitet för medlemsadministration, köp-betalningshantering och bokning av träningsgrupper.
  \item Inpassering via IoT-enheter som styr dörrar och låskomst.
  \item Integrationer mot olika API:er för att automatisera ansökningar om bidrag till ungdomsträningar.
  \item Använder AI-verktyg för att hjälpa användare navigera sidan och snabbt utföra åtgärder.
  \item Designade en skalbar multitenant-arkitektur och uppsatte CI/CD-pipelines för kontinuerlig integration och leverans.
  \item Teknik: C\#, Blazor, .NET MAUI, Azure, IoT, AI, Pipelines (Yaml), Google Cloud.
\end{itemize}

\divider

\cvevent{Utvecklare / Konsult}{Sogeti}{Jan 2023 -- Pågående}

\divider

\cvevent{Fullstack Systemutvecklare / Webbgruppen}{SCA}{Sep 2024 -- Pågående}

\begin{itemize}
  \item Fullstack arbete i ett litet team med fokus på nyutveckling och drift av interna system.
  \item Arbete med drift och incidenthantering.
  \item Arbetade med att sätta upp av CI/CD-pipelines.
  \item Nyutveckling av API:er samt underhåll.  
  \item Integrationer mot externa API:er.  
  \item Arbete med Databas design och drift. Både EF och ren SQL.  
  \item Utveckling av webbapplikationer i MVC (C\#, ASP.NET Core).
  \item Utveckling av webbapplikationer i Angular.
  \item Migrerade projekt från TFS till GitHub. 
  \item Teknik: C\#, .NET Core, .NET Framework, ASP.NET, REST API, SOAP API, MVC, Angular, Git, GitHub, CI/CD (Yaml), GitHub Actions, SQL, SSMS, IIS.
\end{itemize}

\divider

\cvevent{Systemarkitekt \& Utvecklare / Nemus}{SCA}{Nov 2023 -- Sep 2024}

\begin{itemize}
  \item Ansvarade för design av arkitektur och implementering av service i C\# som genererar XML-dokument för avverkningsanmälan.
  \item Utvecklade integration mot externt API.
  \item Designade databasmodeller och hanteringsflöden för validering, lagring och export av data.
  \item Agerade rådgivare till design av verktyg för användarinmatning vilket förenklade arbetsflödet för kundens användare.
  \item Arbetet innebar nära kontakt med kund och externa leverantörer; deltog i kravworkshops och leveransplanering.
  \item Levererad och driftsatt i september 2024 med fortsatt driftstöd och iterativ nyutveckling vid behov.
  \item Teknik: C\#, XML, REST API, ArcGis, PostGIS, SQL.
\end{itemize}

\divider

\cvevent{Systemarkitekt \& Utvecklare}{SCA}{Apr 2023 -- Maj 2023}

\begin{itemize}
  \item Ansvarade för design av arkitektur och implementering av modernisering av ett äldre filhanteringssystem skrivet i Delphi.
  \item Arbetet genomfördes i litet team med nära samarbete med kund.
  \item Levererad och driftsatt i september 2024 med fortsatt driftstöd och iterativ nyutveckling vid behov.
  \item Teknik: C\#, .Net Core, XML, SOAP API.
\end{itemize}

\divider
% \newpage

% \vspace*{0.5cm}

% \cvsection{Arbetslivs Erfarenhet}
% \vspace{0.5cm}

\cvevent{Utvecklare}{Stor Myndighet}{Jan 2023 -- Apr 2023}

\begin{itemize}
  \item Arbetade agilt i ett stort team för att utveckla en administrationswebbplats samt ett utlåningssystem.
  \item Ingick i liten grupp som ansvarade för arkitekturdesign, uppsättning av projektets grundstruktur samt implementering av autentiseringlösning för mikrotjänsterna.
  \item Utveckling av användarrelaterade sidor i Angular-frontenden, inklusive registrering, inloggning och profilsidor.
  \item Utveckling av databasmodeller och integration mot API:er.
  \item Teknik: C\#, Azure, .Net Core, Angular, REST API, Microservices, SQL.
\end{itemize}

% \divider

% \cvevent{Product Engineer}{Google}{23 June 1999 -- 2001}{Palo Alto, CA}

% \begin{itemize}
% \item Joined the company as employe \#20 and female employee \#1
% \item Developed targeted advertisement in order to use user's search queries and show them related ads
% \end{itemize}

% use ONLY \newpage if you want to force a page break for
% ONLY the currentc column

\divider

\cvevent{Utvecklare \& Forskare / Masterarbete}{Mid Sweden University}{Sep 2022 -- Apr 2023}

\begin{itemize}
  \item Proof of concept av video visualiseringsmetod för skytteträning.
  \item Utveckling av visualiseringsverktyg i Python.
  \item Skyttevideos spelades in med en video kamera + Raspberry Pi.
  \item Teknik: IoT, Python, Video Processing
\end{itemize}

\divider

\cvevent{Utvecklare / SIMS}{DIGG}{Sep 2021 -- Dec 2021}

\begin{itemize}
  \item Utvecklade ett Webb API i Python som genererar utvärderad meta-data från öppen data 
  \item Arbete i ett team av studenter med agil produktutveckling.
  \item Teknik: Python, REST API, WebbScraping
\end{itemize}

\divider

\cvevent{Utvecklare / C-Uppsats}{Easit}{Juni 2021 -- Sep 2021}

\begin{itemize}
  \item Arbetade med Integration av en webbapp mot en bot service genom Microsoft Teams
  \item Teknik: C\#, Azure, Microsoft Teams, REST API
\end{itemize}

%% Switch to the right column. This will now automatically move to the second
%% page if the content is too long.
\switchcolumn

\cvsection{Beskrivning}

Erfaren mjukvaruutvecklare och systemarkitekt med över 10 års erfarenhet inom programmering. Har framgångsrikt lett och levererat projekt från kravfångst till produktionssättning, med starkt fokus på skalbar arkitektur, automatiserad CI/CD och mätbart kundvärde. Trivs med att ta helhetsansvar – från att designa robusta system-lösningar till att optimera driftsmiljöer och utvecklingstempo. Söker utmanande uppdrag vid behov av teknisk expertis och affärsnytta.

\cvsection{fritid}

\cvtag{Tennis}
\cvtag{Fiske}
\cvtag{Isbad}\\
\cvtag{Vandra}
\cvtag{Gaming}
\cvtag{Brädspel}

\cvsection{Styrkor}

\cvtag{Problemlösning}
\cvtag{Kommunikation}\\
\cvtag{Hårt Arbetande}
\cvtag{Tekniskt Nyfiken}

\divider\smallskip

\cvtag{Lugn}
\cvtag{Snäll}
\cvtag{Lagspelare}
\cvtag{Proaktiv}

\cvsection{Utbildning}

\cvedu{Civilingenjör i datateknik}{Mittuniversitetet}{Sep 2016 -- Maj 2023}{}

\cvsection{Publikation}

\href{https://urn.kb.se/resolve?urn=urn:nbn:se:miun:diva-48041}{Examensarbete}

\cvsection{Övrig Kunskap}

\cvtag{Svenska}
\cvtag{Engelska}
\cvtag{Körkort}

\cvsection{Certifikat}

\cvcert{AZ-204}{Mars 2023}
\cvcert{AZ-900}{Juli 2023}
\cvcert{AI-900}{Juli 2023}

\newpage

\vspace*{0.5cm}

\cvsection{Kunskap}
\cvevent{}{Programspråk}{}
\cvtag{C\#}
\cvtag{C++}
\cvtag{Javascript}
\cvtag{Typescript}
\cvtag{C}
\cvtag{Python}
\cvtag{Html}
\cvtag{CSS}

\cvevent{}{Webbteknologier \& Ramverk}{}
\cvtag{ASP.NET}
\cvtag{MVC}
\cvtag{Blazor}\\
\cvtag{Angular}
\cvtag{Minimal Api}

\cvevent{}{Moln \& Infrastruktur}{}
\cvtag{Azure}
\cvtag{Google Cloud}
\cvtag{IIS}

\cvevent{}{Verktyg}{}
\cvtag{Git}
\cvtag{Github}
\cvtag{Jira}
\cvtag{Confluence}\\
\cvtag{Visual Studio}
\cvtag{Visual Studio Code}\\
\cvtag{TFS}
\cvtag{AI}
\cvtag{Figma}
\cvtag{Draw.io}

\cvevent{}{Arbetsätt}{}
\cvtag{Clean Architecture}
\cvtag{Scrum}\\
\cvtag{Agile}
\cvtag{Devops}
\cvtag{CI/CD}

\cvevent{}{Plattformar \& tjänster}{}
\cvtag{Microservices}
\cvtag{API}\\
\cvtag{Rest}
\cvtag{Github Actions}

\cvevent{}{Databaser \& databasverktyg}{}
\cvtag{SQL}
\cvtag{Entity Framework (EF)}\\
\cvtag{SSMS}
\cvtag{PostgreSQL}
\cvtag{PostGIS}

\cvevent{}{Övrigt}{}
\cvtag{GIS}
\cvtag{ArcGis}
\cvtag{Nemus}
\cvtag{Skogsvård}
\cvtag{IoT}
\cvtag{Json}
\cvtag{Xml}
\cvtag{Yaml}

\end{paracol}

\end{document}
